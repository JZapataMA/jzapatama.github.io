\documentclass[10pt, letterpaper]{article}

% Packages:
\usepackage[
]{geometry}
\usepackage{titlesec} 
\usepackage{tabularx}
\usepackage{array}
\usepackage[dvipsnames]{xcolor}
\definecolor{primaryColor}{RGB}{0, 0, 0}
\usepackage{enumitem}
\usepackage{fontawesome5}
\usepackage{amsmath}
\usepackage[
    pdftitle={Juanma Zapatas's CV},
    pdfauthor={Juanma Zapata},
    pdfcreator={LaTeX with RenderCV},
    colorlinks=true,
    urlcolor=primaryColor
]{hyperref} % for links, metadata and bookmarks
\usepackage[pscoord]{eso-pic} % for floating text on the page
\usepackage{calc} % for calculating lengths
\usepackage{bookmark} % for bookmarks
\usepackage{lastpage} % for getting the total number of pages
\usepackage{changepage} % for one column entries (adjustwidth environment)
\usepackage{paracol} % for two and three column entries
\usepackage{ifthen} % for conditional statements
\usepackage{needspace} % for avoiding page brake right after the section title
\usepackage{iftex} % check if engine is pdflatex, xetex or luatex
\usepackage{ragged2e}


% Ensure that generate pdf is machine readable/ATS parsable:
\ifPDFTeX
    \input{glyphtounicode}
    \pdfgentounicode=1
    \usepackage[T1]{fontenc}
    \usepackage[utf8]{inputenc}
    \usepackage{lmodern}
\fi

\usepackage{charter}

% Some settings:
\raggedright
\AtBeginEnvironment{adjustwidth}{\partopsep0pt} % remove space before adjustwidth environment
\pagestyle{empty} % no header or footer
\setcounter{secnumdepth}{0} % no section numbering
\setlength{\parindent}{0pt} % no indentation
\setlength{\topskip}{0pt} % no top skip
\setlength{\columnsep}{0.15cm} % set column seperation
\pagenumbering{gobble} % no page numbering

\titleformat{\section}{\needspace{4\baselineskip}\bfseries\large}{}{0pt}{}[\vspace{1pt}\titlerule]

\titlespacing{\section}{
    % left space:
    -1pt
}{
    % top space:
    0.3 cm
}{
    % bottom space:
    0.2 cm
} % section title spacing

\renewcommand\labelitemi{$\vcenter{\hbox{\small$\bullet$}}$} % custom bullet points
\newenvironment{highlights}{
    \begin{itemize}[
        topsep=0.10 cm,
        parsep=0.10 cm,
        partopsep=0pt,
        itemsep=0pt,
        leftmargin=0 cm + 10pt
    ]
}{
    \end{itemize}
} % new environment for highlights


\newenvironment{highlightsforbulletentries}{
    \begin{itemize}[
        topsep=0.10 cm,
        parsep=0.10 cm,
        partopsep=0pt,
        itemsep=0pt,
        leftmargin=10pt
    ]
}{
    \end{itemize}
} % new environment for highlights for bullet entries

\newenvironment{onecolentry}{
    \begin{adjustwidth}{
        0 cm + 0.00001 cm
    }{
        0 cm + 0.00001 cm
    }
}{
    \end{adjustwidth}
} % new environment for one column entries

\newenvironment{twocolentry}[2][]{
    \onecolentry
    \def\secondColumn{#2}
    \setcolumnwidth{\fill, 4.5 cm}
    \begin{paracol}{2}
}{
    \switchcolumn \raggedleft \secondColumn
    \end{paracol}
    \endonecolentry
} % new environment for two column entries

\newenvironment{threecolentry}[3][]{
    \onecolentry
    \def\thirdColumn{#3}
    \setcolumnwidth{, \fill, 4.5 cm}
    \begin{paracol}{3}
    {\raggedright #2} \switchcolumn
}{
    \switchcolumn \raggedleft \thirdColumn
    \end{paracol}
    \endonecolentry
} % new environment for three column entries

\newenvironment{header}{
    \setlength{\topsep}{0pt}\par\kern\topsep\centering\linespread{1.5}
}{
    \par\kern\topsep
} % new environment for the header

\newcommand{\placelastupdatedtext}{% \placetextbox{<horizontal pos>}{<vertical pos>}{<stuff>}
  \AddToShipoutPictureFG*{% Add <stuff> to current page foreground
    \put(
        \LenToUnit{\paperwidth-2 cm-0 cm+0.05cm},
        \LenToUnit{\paperheight-1.0 cm}
    ){\vtop{{\null}\makebox[0pt][c]{
        \small\color{gray}\textit{Last updated in September 2024}\hspace{\widthof{Last updated in September 2024}}
    }}}%
  }%
}%

% save the original href command in a new command:
\let\hrefWithoutArrow\href

% new command for external links:


\begin{document}
    \newcommand{\AND}{\unskip
        \cleaders\copy\ANDbox\hskip\wd\ANDbox
        \ignorespaces
    }
    \newsavebox\ANDbox
    \sbox\ANDbox{$|$}

    \begin{header}
        \fontsize{25 pt}{25 pt}\selectfont Juan Manuel Zapata

        \vspace{5 pt}

        \normalsize
        \mbox{Santiago, CL}%
        \kern 5.0 pt%
        \AND%
        \kern 5.0 pt%
        \mbox{\hrefWithoutArrow{mailto:juanmazm9@gmail.com}{juanmazm9@gmail.com}}%
        \kern 5.0 pt%
        \AND%
        \kern 5.0 pt%
        \mbox{\hrefWithoutArrow{tel:+569 4015 2597}{9 4015 2597}}%
        \kern 5.0 pt%
        \AND%
        \kern 5.0 pt%
        \mbox{\hrefWithoutArrow{https://linkedin.com/in/juanma-zapatam}{linkedin.com/in/juanma-zapatam}}%
        \kern 5.0 pt%
        \AND%
        \kern 5.0 pt%
        \mbox{\hrefWithoutArrow{https://github.com/JZapataMA}{github.com/JZapataMA}}%
        \kern 5.0 pt
        \AND
        \kern 5.0 pt
        \mbox{\hrefWithoutArrow{https://jzapatama.github.io/}{Portafolio}}
    \end{header}

    \vspace{5 pt - 0.3 cm}


\section{Resumen Profesional}

\begin{onecolentry}
\justifying
Profesional con estudios en Ingeniería en Ciencia de Datos, con excelentes habilidades comunicacionales y actitud proactiva, beneficiando en gran medida tanto el trabajo individual como grupal. Especializado en el desarrollo de modelos de \textbf{Machine Learning} y \textbf{Deep Learning}, así como en el tratamiento y análisis de \textbf{datos masivos} con enfoque en escalabilidad. Poseo un sólido entendimiento de la teoría del aprendizaje automático, conocimientos avanzados en \textbf{estadística} y amplia experiencia en la implementación de soluciones basadas en modelos \textbf{LLM} y técnicas de procesamiento de datos. Mi trayectoria abarca desde la exploración y optimización de algoritmos hasta la generación de \textbf{insights accionables} para resolver problemas complejos en diversos contextos.
\end{onecolentry}

\section{Conocimientos}

\begin{onecolentry}
    \begin{highlightsforbulletentries}

\begin{itemize}
    \item \textbf{Inglés} nivel avanzado.
    \item Desarrollo de modelos de \textbf{aprendizaje automático} y \textbf{profundo} con \textbf{TensorFlow}, \textbf{Scikit-learn} y \textbf{PyTorch}.
    \item Dominio avanzado de \textbf{Python}, \textbf{R} y bibliotecas como \textbf{Pandas} y \textbf{NumPy} para análisis y manipulación de datos.
    \item Conocimientos sólidos en \textbf{estadística bayesiana} y \textbf{estructuras de datos}.
    \item Procesamiento de grandes volúmenes de datos con utilización de Pyspark.
    \item Diseño de \textbf{sistemas recomendadores (RecSys)} y algoritmos personalizados para grandes conjuntos de datos.
    \item Manejo de bases de datos relacionales con \textbf{PostgreSQL} para diseño, consulta y optimización.
    \item Competencia en \textbf{programación de bajo nivel} con \textbf{C/C++} y análisis basado en grafos.
    \item Control de versiones con \textbf{Git} y colaboración en proyectos con \textbf{GitHub}.
\end{itemize}
\end{highlightsforbulletentries}
\end{onecolentry}

\section{Educación}

\begin{twocolentry}{
    2021 – 2024
}
    \textbf{Pontificia Universidad Católica de Chile}, Licenciatura en Ingeniería en Ciencia de Datos
\end{twocolentry}

\section{Experiencia}

\begin{twocolentry}{Oct 2023 – Mar 2025}
    \textbf{AI Analyst}, Gather Consultores -- Santiago, CL
\end{twocolentry}

\vspace{0.10 cm}
\begin{onecolentry}
    \begin{highlights}
    \item Responsable de desarrollar aplicaciones integradas que emplean modelos de \textbf{aprendizaje automático} para tareas de \textbf{clasificación}, \textbf{detección}, \textbf{recomendación} y \textbf{generación}, creando herramientas que optimizan recursos y software, agilizando procesos diarios.
    \item Participación en diversos eventos y capacitaciones, entre los cuales se destaca el \textbf{IBM TechxChange 2024}, celebrado en Las Vegas.
    \end{highlights}
\end{onecolentry}

\section{Proyectos}

\begin{twocolentry}{
}
    \textbf{Identificación y Reconocimiento de Asentamientos Informales}
\end{twocolentry}

\vspace{0.10 cm}
\begin{onecolentry}
    \begin{highlights}
    \item Uso de imágenes satelitales de baja calidad para identificar asentamientos informales.
    \item Clasificación de manzanas censales mediante agrupaciones para determinar calificaciones de sectores y posibles expansiones a través del uso de redes neuronales con \textbf{Pytorch} y \textbf{Scikit-learn}.
    \end{highlights}
\end{onecolentry}
\vspace{0.2 cm}
\begin{twocolentry}{
}

    \textbf{Análisis y Búsqueda de Pares de Usuarios}
\end{twocolentry}

\vspace{0.10 cm}
\begin{onecolentry}
    \begin{highlights}
    \item Procesamiento de una base de datos de tweets relacionados con el proyecto de nueva constitución de 2020 en Chile.
    \item Desarrollo de un algoritmo de búsqueda para identificar y clasificar usuarios con publicaciones similares a través del uso de \textbf{BigQuery} y técnicas de \textbf{Local Sensitive Hashing}.
    \end{highlights}
\end{onecolentry}
\vspace{0.2 cm}

    \textbf{Identificación de Duplicidad de Recursos Humanos}
\vspace{0.10 cm}
\begin{onecolentry}
    \begin{highlights}
    \item Diseño de un esquema algorítmico para identificar proyectos duplicados en una institución pública.
    \item Optimización de criterios de evaluación en procesos de cofinanciamiento, mejorando la precisión y eficiencia del análisis de postulaciones con \textbf{Pandas} y \textbf{Numpy}.
    \end{highlights}
\end{onecolentry}
\vspace{0.2 cm}

    \textbf{Sistema de recomendación de canciones en base de letras}
\vspace{0.10 cm}
\begin{onecolentry}
    \begin{highlights}
    \item Procesamiento y vectorización a través de \textbf{transformers} de variables string para generar representaciones de canciones.
    \item Desarrollo algorítmico para un sistema de predicción de items a consumir dado un listado utilizando \textbf{Pytorch}, \textbf{Scikit-learn} y modelos disponibilizados en \textbf{Hugging Face}.
    \end{highlights}
\end{onecolentry}

\end{document}